%----------------------------------------------------------------------------------------
%	PREÁMBULO
%----------------------------------------------------------------------------------------

\documentclass[11pt, oneside]{book}
\usepackage[paperwidth=17cm, paperheight=22.5cm, bottom=2.5cm, right=2.5cm]{geometry}

% El borde inferior puede parecerles muy amplio a la vista. Les recomiendo hacer una prueba de impresión antes para ajustarlo

\usepackage{amssymb,amsmath,amsthm} % Símbolos matemáticos
\usepackage[spanish]{babel}
\usepackage[utf8]{inputenc} % Acentos y otros símbolos 
\usepackage{enumerate}
\usepackage{hyperref} % Hipervínculos en el índice
\usepackage{graphicx}
%\usepackage{subfig} % Subfiguras
\graphicspath{{Imagenes/}} % En qué carpeta están las imágenes

% Para eliminar guiones y justificar texto
\tolerance=1
\emergencystretch=\maxdimen
\hyphenpenalty=10000
\hbadness=10000

\linespread{1.25} % Asemeja el interlineado 1.5 de Word

\let\oldfootnote\footnote % Deja espacio entre el número del pie de página y el inicio del texto
\renewcommand\footnote[1]{%
\oldfootnote{\hspace{0.05mm}#1}}

\renewcommand{\thefootnote} {\textcolor{Black}{\arabic{footnote}}} % Súperindice a color negro

\setlength{\footnotesep}{0.75\baselineskip} % Espaciado entre notas al pie

\usepackage{fnpos} % Footnotes al final de pág.

\usepackage[justification=centering, font=bf, labelsep=period, skip=5pt]{caption} % Centrar captions de tablas y ponerlas en negritas

\newcommand{\imagesource}[1]{{\footnotesize Fuente: #1}}

\usepackage{tabularx} % Big tables
\usepackage{graphicx}
\usepackage{adjustbox}
\usepackage{longtable}

\usepackage{float} % Float tables

\usepackage[usenames,dvipsnames]{xcolor} % Color

\usepackage{pgfplots} % Gráficas
\pgfplotsset{compat=newest}
\pgfplotsset{width=7.5cm}
\pgfkeys{/pgf/number format/1000 sep={}}

\begin{document}

%----------------------------------------------------------------------------------------
%	PORTADA
%----------------------------------------------------------------------------------------

\title{Metodología para diseño de grupos en A/B testing usando Componantes Principales} % Con este nombre se guardará el proyecto en writeLaTex

\begin{titlepage}
\begin{center}

\textsc{\Large Instituto Tecnológico Autónomo de México}\\[2em]

%Figura
\begin{figure}[h]
\begin{center}
\includegraphics[scale=0.50]{itam_logo.png}
\end{center}
\end{figure}

% Pueden modificar el tamaño del logo cambiando la escala

\textbf{\LARGE Metodología para diseño de grupos en A/B testing usando Componentes Principales}\\[2em]

\textsc{\large Tesis}\\[1em]

\textsc{\large que para obtener el título de}\\[1em]

\textsc{\LARGE Licenciado en Matemáticas Aplicadas}\\[1em]

\textsc{\large Presenta}\\[1em]

\textsc{\LARGE Pablo Alonso Sandoval Mendoza}\\[1em]

\textsc{\large Asesor}\\[1em]

\textsc{\LARGE }\\[2em]

% Asegúrense de escribir el nombre completo de su asesor

\end{center}

\vspace*{\fill}
\textsc{Ciudad de México \hspace*{\fill} 2020}

\end{titlepage}

%----------------------------------------------------------------------------------------
%	DECLARACIÓN
%----------------------------------------------------------------------------------------

\thispagestyle{empty}

\vspace*{\fill}
\begingroup

\noindent
«Con fundamento en los artículos 21 y 27 de la Ley Federal del Derecho de Autor y como titular de los derechos moral y patrimonial de la obra titulada ``\textbf{Metodología para diseño de grupos en A/B testing}'', otorgo de manera gratuita y permanente al Instituto Tecnológico Autónomo de México y a la Biblioteca Raúl Bailléres Jr., la autorización para que fijen la obra en cualquier medio, incluido el electrónico, y la divulguen entre sus usuarios, profesores, estudiantes o terceras personas, sin que pueda percibir por tal divulgación una contraprestación.»

% Asegúrense de cambiar el título de su tesis en el párrafo anterior

\centering 

\vspace{5em}

\rule[1em]{20em}{0.5pt} % Línea para la fecha

\textsc{Fecha}
 
\vspace{8em}

\rule[1em]{20em}{0.5pt} % Línea para la firma

\textsc{Pablo Alonso Sandoval Mendoza}

\endgroup
\vspace*{\fill}

%----------------------------------------------------------------------------------------
%	DEDICATORIA
%----------------------------------------------------------------------------------------

\pagestyle{plain}
\frontmatter

\chapter*{}
\begin{flushright}
\textit{A mis padres,\\ por su incansable esfuerzo.}
\end{flushright}

%----------------------------------------------------------------------------------------
%	AGRADECIMIENTOS
%----------------------------------------------------------------------------------------

\chapter*{Agradecimientos}

\noindent Lorem ipsum dolor sit amet, consectetur adipiscing elit.

% Esta sección es lo único que la gente lee. True story :)

%----------------------------------------------------------------------------------------
%	RESUMEN
%----------------------------------------------------------------------------------------

\chapter*{Resumen}

\noindent La implementación de nuevas funcionalidades dentro del desarrollo de un producto puede resultar costoso al no tener resultados esperados y con altos costos de desarrollo para agregar nuevas funcionalidades. A menudo, las funcionalidades se van liberando gradualmente tratando de medir el efecto de éstas con métricas determinadas que ayudan a establecer la viabilidad de la implementación para el universo de usuarios del producto. 
En la actualidad existen diversas metodologías para la selección de primeros usuarios con el fin de medir su comportamiento bajo los cambios hechos. En este trabajo se explora una metodología basada en técnicas estadísticas teniendo como meta la creación de un grupo experimental y un grupo de control con características similares para obtener una mejor comparación de las métricas relevantes para quien implementa la prueba.

\pagestyle{plain}

\noindent 

%----------------------------------------------------------------------------------------
%	Summary
%----------------------------------------------------------------------------------------

\chapter*{Summary}

\noindent The implementation for new functionalities regarding the development of a product could be expensive and not getting the expected results falling on high development costs with these new functionalities. In common practice, functionalities are released to a few group of users to test the effect and viability of the implementation for the whole universe of users.\\
Different methodologies are used to select the first users to get the new functionalities implementation; this work explores a methodology based in statistical techniques which have the goal to create an experimental and control group with similar statistic characteristics in order to get a better comparison between them. 

\pagestyle{plain}

\noindent 

%----------------------------------------------------------------------------------------
%	TABLA DE CONTENIDOS
%---------------------------------------------------------------------------------------

\tableofcontents

%----------------------------------------------------------------------------------------
%	ÍNDICE DE CUADROS Y FIGURAS
%---------------------------------------------------------------------------------------

\listoftables

\listoffigures

%----------------------------------------------------------------------------------------
%	TESIS
%----------------------------------------------------------------------------------------

\mainmatter % Empieza la numeración de las páginas

\pagestyle{plain}

% Incluye los capítulos en el fólder de capítulos

\include{Chapters/0.Intro}

\chapter{Revisión de literatura}

\noindent   

\newpage

\section{A/B testing}

\subsection{Estado del arte}

\section{Análisis de Componentes Principales}
\noindent El Análisis de Componentes Principales (PCA por sus iniciales en inglés) es una técnica para describir un conjunto de variables transformándolas en nuevos sistemas de coordenadas utilizando álgebra lineal. Esta transformación en un nuevo sistema está dado por la varianza de los datos y por las correlaciones entre ellos. Las componentes principales son el nuevo sistema de coordenadas.
\subsection{Álgebra Lineal}
\subsection{PCA}







\chapter{Análisis de Componentes Principales}

\noindent Explicar PCA

\newpage

\section{Älgebra Lineal}

\section{PCA}
% Para diseñar tablas


\chapter{Bases de datos}

\noindent Construcción de la base de datos

\newpage

\section{Supuestos}



\chapter{Metodología}

\noindent explicar metodología


\chapter{Resultados}
\noindent Gráficas de comparación de variables entre los grupos
\sction{Distribución de variables continuas}


\chapter*{Conclusiones}
\addcontentsline{toc}{chapter}{Conclusiones}

% Las conclusiones tampoco cuentan como capítulo

\noindent Hegel (1953) escribe en su obra \textit{Lecciones sobre la filosofía de la Historia Universal} un par de conceptos que son útiles para comprender de manera abstracta la esencia detrás de esta investigación.


%----------------------------------------------------------------------------------------
%	APÉNDICES
%----------------------------------------------------------------------------------------

\begin{appendix}

\include{Apendices/ApA}

\end{appendix}

%----------------------------------------------------------------------------------------
%	BIBLIOGRAFÍA
%----------------------------------------------------------------------------------------


\chapter*{Referencias}
\addcontentsline{toc}{chapter}{Referencias}

% Macro. Esto es muy importante, no lo borren

\makeatletter
\renewenvironment{thebibliography}[1]
     {\@mkboth{\MakeUppercase\refname}{\MakeUppercase\refname}%
      \list{}%
           {\setlength{\labelwidth}{0pt}%
            \setlength{\labelsep}{0pt}%
            \setlength{\leftmargin}{\parindent}%
            \setlength{\itemindent}{-\parindent}%
            \@openbib@code
            \usecounter{enumiv}}%
      \sloppy
      \clubpenalty4000
      \@clubpenalty \clubpenalty
      \widowpenalty4000%
      \sfcode`\.\@m}
     {\def\@noitemerr
       {\@latex@warning{Empty `thebibliography' environment}}%
      \endlist}
\makeatother

\begin{thebibliography}{111}

% Lista

% La manera recomendada para citar papers o libros en el formato de Chicago esta en el siguiente vínculo: https://www.chicagomanualofstyle.org/tools_citationguide/citation-guide-2.html

% Es importante poner el apellido del autor seguido del año de publicación, una coma y las páginas consultadas en el texto antes de puntuar y entre paréntesis para las citas en el cuerpo de la tesis

% Ejemplo:

% Las \textit{causas próximas} del crecimiento son conocidas: tecnología, capital humano y físico. La pregunta es ¿por qué unos países sí tienen estas causas próximas y otros no? La respuesta son las \textit{causas fundamentales:} suerte, geografía, cultura e instituciones (Acemoglu 2009, 110).

%AAAAA
\bibitem{abram57} Abramovitz, Moses. 1957. «Resources on Output Trends in the United States since 1870.» \textit{The American Economic Review} 46 (2): 5–23.

\bibitem{acemoglu09} Acemoglu, Daron. 2009. \textit{Introduction to Modern Economic Growth.} Princeton: Princeton University Press.

%BBBBB

%CCCCC

%DDDDD

%EEEEE

%FFFFF

%GGGGG

%HHHHH

%IIIII

%JJJJJ

%KKKKK

%LLLLL

%MMMMM

%NNNNN

%OOOOO

%PPPPP

%QQQQQ

%RRRRR

%SSSSS

%TTTTT

%UUUUU

%VVVVV

%WWWWW

%XXXXX

%YYYYY

%ZZZZZ

\end{thebibliography}


\end{document}