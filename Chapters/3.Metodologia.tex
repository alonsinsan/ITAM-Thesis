\chapter{Metodología}

\noindent Dada la naturaleza de los Componentes Principales obtenidos, sabemos que en la primera componente principal se acumula la mayor varianza; en el 2o Componente Principal la segunda mayor varianza y así sucesivamente con las demás COmponentes hasta llegar a $m$ que es igual al número de variables en nuestro conjunto de datos. La metodología busca aprovechar esta característica para la división de grupos dividiendo en cuartiles sobre cada Componente Principal y obteniendo muestras aleatorias de cada uno de estos nuevos conjuntos. Escogiendo los grupos de Control y Experimental de esta forma, la mediana y los cuartiles serán parecidos entre ellos.
Se hace una prueba de hipótesis para comparar las medias de estas poblaciones y encontramos cuál es la región de rechazo.
\section{Paso a paso}
\begin{enumerate}
    \item Validar correlación de variables y eliminar variables correlacionadas.
    \item Obtener primeras dos componentes principales con los respectivos eigenvalores.
    \item Varianza acumulada en primeras dos componentes
    \item Transformar variables al nuevo espacio $\mathbb{R}^2$ con los ejes igual a las componentes principales.
    \item Obtener cuartiles de la distribución conjunta de los datos en el nuevo sistema de coordenadas.
    \item Para cada cuartil, tomar una muestra aleatoria de índices dentro del cuartil con proporción 80-20. (80\% grupo de control-20\% grupo experimental)
    \item Unir los grupos obtenidos en cada cuartil para tener el grupo completo experimental y el grupo completo de control.
    \item Prueba de hipótesis de comparación de rango intercuartílico.
\end{enumerate}
\section{Código}
\begin{itemize}
    \item INPUT: 
    \begin{itemize}
        \item Archivo csv con las variables a explorar. Columna con ID's.
        \item Input script:
        \begin{itemize}
            \item nombre\_indice: nombre de la columna que es índice.
        \end{itemize}
    \end{itemize}
    \item Funciones
    \lstinputlisting[language=Python, caption=Transformación PCA]{Code/transform_pca.py}
    \lstinputlisting[language=Python, caption=Creación de cuantiles y etiquetado]{Code/make_groups.py}
    \lstinputlisting[language=Python, caption=División iterando en cada cuartil]{Code/iterate_division.py}
    \lstinputlisting[language=Python, caption=Main]{Code/main.py}
    
\end{itemize}