\chapter{Revisión de literatura}

\noindent   

\newpage

\section{A/B testing}

\subsection{Estado del arte}

\section{Análisis de Componentes Principales}
\noindent El Análisis de Componentes Principales (PCA por sus iniciales en inglés) es una técnica para describir un conjunto de variables transformándolas en nuevos sistemas de coordenadas utilizando álgebra lineal. Esta transformación en un nuevo sistema está dado por la varianza de los datos y por las correlaciones entre ellos. Las componentes principales son el nuevo sistema de coordenadas.
\subsection{Álgebra Lineal}
\subsection{PCA}





